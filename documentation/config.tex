% !TEX root =  master.tex
% 		HYPERREF
%
%%%%%]{hyperref}

%		FONT AND INPUT ENCODING 
%
\usepackage[T1]{fontenc}
\usepackage[utf8]{inputenc}
\usepackage{setspace}
\onehalfspacing


%		CALCULATIONS 
%
\usepackage{calc} % Used for extra space below footsepline

%		LANGUAGE SETTINGS
%
\usepackage[ngerman]{babel} 	% German language 
\usepackage[german=quotes]{csquotes} 	% correct quotes using \enquote{}  

%\usepackage[english]{babel}   % For english language
%\usepackage{csquotes} 	% Richtiges Setzen der Anführungszeichen mit \enquote{}  


%		BIBLIOGRAPHY SETTINGS
%
 \usepackage[backend=biber, autocite=footnote, style=authoryear-icomp, dashed=false]{biblatex} 	%Use Author-Year-Cites with footnotes 
% \usepackage[backend=biber, autocite=inline, style=ieee]{biblatex} 	% Use IEEE-Style (e.g. [1]) 
%\usepackage[backend=biber, autocite=inline, style=alphabetic]{biblatex} 	% Use alphabetic style (e.g. [TGK12]) 
% \usepackage[backend=biber, autocite=inline, style=authoryear, dashed=false]{biblatex} 	% Harvard-Style


\DefineBibliographyStrings{ngerman}{  %Change u.a. to et al. (german only!)
	andothers = {{et\,al\adddot}},             
} 

\setlength{\bibparsep}{\parskip}		%add some space between biblatex entries in the bibliography
\addbibresource{bibliography.bib}	%Add file bibliography.bib as biblatex resource 


%	ACRONYMS
%%% 
%%% WICHTIG: Installieren Sie das neueste Acronyms-Paket!!! 
%%% 
\makeatletter
\usepackage[printonlyused]{acronym}
\@ifpackagelater{acronym}{2015/03/20}
  {%
    \renewcommand*{\aclabelfont}[1]{\textbf{\textsf{\acsfont{#1}}}}  
  }%
  {%
  }%
\makeatother 



%		LISTINGS
\usepackage{listings}	%Format Listings properly 

\usepackage{color}
\renewcommand{\lstlistlistingname}{Quelltextverzeichnis}
\renewcommand{\lstlistingname}{Quelltext}


\definecolor{hpegreen}{RGB}{1,169,130}
\definecolor{hpegray1}{RGB}{198,201,202}
\definecolor{hpeorange}{RGB}{255,141,109}
\definecolor{hpeturquoise}{RGB}{42,210,201}
\definecolor{gray1}{RGB}{245,245,245}

\lstset{ %
	backgroundcolor=\color{gray1}, 
	basicstyle=\footnotesize,             
	breakatwhitespace=false,              
	breaklines=true,                            
	captionpos=b,                 				 
	commentstyle=\color{hpegreen}, 
	deletekeywords={...},                   
	escapeinside={(*@}{@*)}, 		   
	extendedchars=true,            			
	keepspaces=true,                			 
	keywordstyle=\color{hpeturquoise},      		
	otherkeywords={*,...},           			
	numbers=left,                   
	numbersep=10pt,                   
	numberstyle=\tiny\color{hpegray1}, 
	rulecolor=\color{black},       
	showspaces=false,               
	showstringspaces=false,        
	showtabs=false,                 
	stringstyle=\color{hpeorange},    
	tabsize=3,	                   
	title=\lstname                   
}



\renewcommand{\lstlistlistingname}{Quelltextverzeichnis}
\lstset{numbers=left,
	numberstyle=\tiny,
	captionpos=b,
	basicstyle=\ttfamily\small}


%		EXTRA PACKAGES 
\usepackage{lipsum}    %Blindtext
\usepackage{graphicx} % use various graphics formats  
\usepackage[german]{varioref} 	% nicer references \vref
\usepackage{caption}	%better Captions 
\usepackage{booktabs} %nicer Tabs



%		ALGORITHMS 
\usepackage{algorithm}  
\usepackage{algpseudocode}
\renewcommand{\listalgorithmname}{Algorithmenverzeichnis }
\floatname{algorithm}{Algorithmus}


%		FONT SELECTION: Entweder Latin Modern oder Times / Helvetica
\usepackage{lmodern} %Latin modern font 
%\usepackage{mathptmx}  %Helvetica / Times New Roman fonts (2 lines)
%\usepackage[scaled=.92]{helvet} %Helvetica / Times New Roman fonts (2 lines)

%		PAGE HEADER / FOOTER 
%	    Warning: There are some redefinitions throughout the master.tex-file!  
\RequirePackage[automark,headsepline,footsepline]{scrpage2}
\pagestyle{scrheadings}
\renewcommand*{\pnumfont}{\upshape\sffamily}
\renewcommand*{\headfont}{\upshape\sffamily}
\renewcommand*{\footfont}{\upshape\sffamily}
\renewcommand{\chaptermarkformat}{}

\clearscrheadfoot

\ifoot[\rule{0pt}{\ht\strutbox+\dp\strutbox}DHBW Mannheim]{\rule{0pt}{\ht\strutbox+\dp\strutbox}DHBW Mannheim}
\ofoot[\rule{0pt}{\ht\strutbox+\dp\strutbox}\pagemark]{\rule{0pt}{\ht\strutbox+\dp\strutbox}\pagemark}

\ohead{\headmark}





